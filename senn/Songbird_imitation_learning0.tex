\documentclass[10pt]{article}
\usepackage{amssymb,amsmath}
\usepackage{graphicx}
\usepackage{cite}
\usepackage[usenames,dvipsnames]{color}
%\usepackage{/users/senn/Documents/Walter_Physio/Bibfiles/jneurosci}
\usepackage{jneurosci}



\usepackage{todonotes}
\newcommand{\ag}[1]{\todo[inline,color=yellow!40]{#1}}
\newcommand{\ws}[1]{\todo[inline,color=blue!15]{#1}}
\newcommand{\rh}[1]{\todo[inline,color=blue!15]{#1}}



\topmargin 0.0cm
\oddsidemargin 0.5cm
\evensidemargin 0.5cm
\textwidth 16cm 
\textheight 21cm

% Bold the 'Figure #' in the caption and separate it with a period
% Captions will be left justified
\usepackage[labelfont=bf,labelsep=period,justification=raggedright]{caption}
\DeclareGraphicsExtensions{.eps}

\pagestyle{myheadings}
\newcommand{\RA}{\mathrm{RA}}
\newcommand{\AC}{\mathrm{AC}}
\newcommand{\dtau}{\mathrm{\!\!\!\!\!\! d\tau}}
\newcommand{\LMAN}{\mathrm{LMAN}}
\newcommand{\HVC}{\mathrm{HVC}}
\newcommand{\oLMAN}{\overline{\mathrm{LMAN}}}
\newcommand{\oHVC}{\overline{\mathrm{HVC}}}
\newcommand{\cP}{ {\cal{P}} }
\newcommand{\SPL}{\mathrm{SPL}}
\newcommand{\mn}{\mathrm{mean\,}}
\newcommand{\fthr}{\mathrm{tut}}
\newcommand{\MEMTUT}{\mathrm{AUDTUT}}
\newcommand{\one}{\mathbb{1}}

\begin{document}
\begin{flushleft}
{\LARGE
  \textbf{\color{red}{Songbird imitation learning}}
}
\end{flushleft}

\noindent
Walter Senn\footnote{Department of Physiology, University of Bern, senn@pyl.unibe.ch, +41 31 631 87 21}, Andr\'e G\"uning\footnote{University of Surrey, a.gruning@ac.uk}, Richard Hahnloser\footnote{Institute of Neuroinformatics, ETHZ/UNIZH, rich@ini.ethz.ch}\\[1ex] \today 

\subsection*{Model of song generation}
%======================================
Consider the sound pressure time course $p(t)$ of a (tutor or own)
song and consider the Fourier decomposition of the time-continuous
sound pressure level $\SPL(t) = \log \lfloor p(t)/p_\circ \rfloor$,
with $\lfloor \cdot \rfloor$ representing the Heaviside function (ie
cut-off at zero).
\begin{equation}
\SPL(t) \approx  \sum_i^N  \left( \phi^s_i(t) \sin(\omega_i t) + \phi^c_i(t) \cos(\omega_i t) \right) \,,
\label{SPL}
\end{equation}
for appropriate frequencies $\omega_i$.

LMAN neurons receive conductance-based inputs from HVC and from a lower auditory area representing the Fourier components $\phi_i$. To represent the full auditory signal with non-negative firing rates, we consider $4N$ neurons in LMAN with activity $\LMAN^\nu_i(t)$ specified at each discrete time step $t$, with $i=1\dots N$ and $\nu \in \{ s+,s-, c+,c- \}$ specifying the positive and negative parts of the sine and cosine contribution to the sound pressure level in Eq.\ \ref{SPL}. As time step we choose $dt=10\,$ms, the rough duration of an HVC burst. To fully specify the LMAN activity we first need to describe HVC.

HVC is modelled as a delay line where each neuron fires a $10\,$ms
burst once during the whole song. If the song duration is $T$ times
$10\,$ms, we need $T$ neurons to cover the whole song duration with
HVC activity. These neurons are then active at one single time step
$t=1\dots T$ during the song, $\HVC^i(t) = \delta_{it}$, for $i=1\dots
T$. The tutor song is stored in the synaptic strengths $w_{\HVC \to
  \LMAN}$ from HVC to LMAN. Let us denote the synaptic strength from
HVC neuron $i$ to LMAN neuron $(\nu,j)$ by $w_{{\HVC^i} \to
  {\LMAN^\nu_j}}$. \ag{I agree -- my python code models this, however
  currently without the $s+\dots c-$ split -- how about WS' Matlab code?}
.

We can now define the time course of the LMAN neurons according to
\begin{eqnarray}
\LMAN^{s+}_i(t) & = &  \mu \, w_{{\HVC^t} \to {\LMAN^{s+}_i }}  +   (1-\mu) \, \lfloor \overline{\phi^s_i(t\,dt)}  \rfloor \nonumber \\
\LMAN^{s-}_i(t) & = & \mu \, w_{{\HVC^t} \to {\LMAN^{s-}_i }} + (1-\mu) \, \lfloor -\overline{\phi^s_i(t\,dt)}  \rfloor  \,,
\label{LMAN}
\end{eqnarray} where the mean $\overline{\phi^\nu_i(t\,dt)}$ is taken across the interval of duration $dt$ starting at time $t\,dt$ for the discrete times steps $t=0,1,\dots, T$. The activities $\LMAN^{c\pm}_i$ representing the positive and negative contribution of the cosine in (\ref{SPL}) are defined correspondingly. 

\subsubsection{Phase A: Imprinting of Tutor Song}

The synaptic strengths from HVC are set to 
\begin{equation}
w_{{\HVC^i} \to {\LMAN^\nu_j}} = \LMAN^\nu_j(t) \,, \; \mbox{ for }  t=i
\label{wHVCLMAN}
\end{equation}
\ag{Ought to be
$%\begin{equation}
w_{{\HVC^i} \to {\LMAN^\nu_j}} = \LMAN^\nu_j(t-\Delta) \,, \; \mbox{ for }  t=i
$%\end{equation}
because LMAN is always $\Delta$ behind HVC, compare \eqref{i2} and \eqref{i3}.
}
where LMAN activity is recorded during the tutor song presentation, $p(t)=p^\mathrm{tutor}(t)$, while the memory input from HVC to LMAN is suppressed ($\mu=0$).

RA neurons are working in two modes. They either exclusively integrate
input from LMAN while adapting the $\HVC \to \RA$ synapse (`weight
copying', predictive learning rule, $\lambda=0$), or they exclusively
integrate input from HVC while adapting the $\HVC \to \RA$ synapse
\ag{Should be: $\LMAN \to RA$} (`inverse model learning', postdictive learning rule, $\lambda=1$). Formally, the activity of RA neuron $j=1\dots N_\RA$ is
\begin{equation}
\RA_j (t)  = \lfloor \lambda \, \overline{\HVC}_j (t) + (1-\lambda) \, \overline{\LMAN}_j (t)  \rfloor \,, 
\label{RA}
\end{equation}
where $\lfloor . \rfloor$ again represents the hard cut off at 0 to mimic the threshold-linear transfer function. Because at time step $t$ only the single HVC neuron $\HVC^i$ with $i=t$ is active, $\HVC^i(t)=1$, the total input from HVC to RA at time $t$ is identical to the synaptic weight,
\begin{equation}
\overline{\HVC}_j (t)  = w_{{\HVC^i} \to {\RA_j}} \,, \; \mbox{ for }  i=t \,.
\label{HVCRA}
\end{equation}
The total LMAN input to RA neuron $j$ is given by
\begin{equation}
\overline{\LMAN}_j (t)  = \sum_{i,\nu}^{N,4} w_{{\LMAN^\nu_i} \to {\RA_j}} \,  \LMAN^\nu_i (t) \,.
\label{LMANRA}
\end{equation}

The song is specified by a linear mixture of the RA activities. Let $M$ be the mixing matrix that converts the $\RA_j$ activities ($j=1 \dots N_\RA$) to $N$ pairs of frequency coefficients $(\psi_i^s, \psi_i^c)$, $i=1\dots N$, with $\psi = M\,\RA$. Note that the $2N$ components of $\psi$ are evaluated at each discrete time step $t=1\dots T$. The entries of $M$ may be sampled from a Gaussian distribution around $0$. The song that is produced by this mixture is defined by the sound pressure level
\begin{equation}
\SPL_\psi (t) =  \sum_i^N  \left( \psi^s_i( \lfloor t \rfloor_{dt} ) \sin(\omega_i t ) + \psi^c_i( \lfloor t \rfloor_{dt} ) \cos(\omega_i t ) \right) \,,
\label{SPLgen}
\end{equation}
where $t$ is here again the continuous time variable and $\lfloor t \rfloor_{dt} = \mathrm{floor}(t/dt)$ is the index of the $dt$ time bin containing $t$.  The continuous sound pressure is obtained by $p(t) = p_\circ e^{\SPL_\psi (t)}$. 

As a reminder: for RA activity traces see \citetext{Leonardo2005} and for HVC activity traces see \citetext{Kozhevnikov2007}.

\subsection*{Functional plasticity rules for RA-projecting synapses}
%======================================
Based on our scheme for imitation learning in the zebra finch we
consider a predictive learning for HVC$\to$RA synapses and a
postdictive learning for LMAN$\to$RA synapses. 

\subsubsection{Phase B -- activity copying}

From a functional point of view, the predictive HVC$\to$RA  plasticity should have the form
\begin{equation}
\triangle w_{\HVC \to \RA}(t)  = \left(\RA_t -  \oHVC_{t-\Delta} \right) \, \HVC_{t-\Delta} \,,
\label{HVC-RA}
\end{equation}
where $\Delta$ represents the motor-to-auditory delay, roughly
$70\,$ms. For notational convenience we suppressed the neuron indices
and instead write time as index. Moreover, $\HVC_{t-\Delta}$ is the
activity of the specific presynaptic HVC neuron at time $t-\Delta$,
and $\oHVC_{t-\Delta}$ is the total RA drive stemming from all HVC
neurons. This rule is active while the RA neuron is actually driven by
the LMAN input ($\lambda = 0)$). The idea behind the difference
learning is that in a later stage, when RA is driven by HVC, the same
RA activity can be reproduced by HVC, albeit now advanced by $\Delta$.  

\subsubsection{Phase C -- causal inverse learning}

The postdictive LMAN$\to$RA plasticity should have the form 
\begin{equation}
\triangle w_{\LMAN \to \RA}(t)  = \left(\RA_{t-\Delta} -  \oLMAN_t \right) \, \LMAN_t \,.
\label{LMAN-RA}
\end{equation}


\subsubsection{Justification of Rule? Or to deal with intervening spikes?}

\ag{Not clear to me what the purpose of these paragraphs is. What do
  we want to demonstrate here? Justify our rule better? Explore how our
  rule reproduces the graph from Doupe? Argue that we should not
  have intervening spikes between 2 activations of RA neurons
  less than $\Delta$ apart? Or part of argumentation how the
  eligibility trace for post/predictive learning might be implemented?} 

The RA activity (as the presynaptic activities) is intrinsically
stochastic. If at time $t$ the RA neuron is spiking, this may only be by chance. A better learning rule is obtained by averaging out the stochasticity. Assuming that the RA activity at a given point in time is 1 or 0, we replace $\RA_t$ in (\ref{HVC-RA}) by its expectation that then takes the form
\begin{equation}
\langle \RA_t \rangle  = \cP \{ \RA_t = 1 \, | \, \RA_{t'<t}  \} \approx  \int_0^{\infty} \dtau \, \AC_\RA(\tau) \RA(t-\tau)  \,
\label{expRA}
\end{equation}
where $\AC_\RA$ represents the auto-correlation function of the RA
activity. Because after the RA neuron is typically bursting, $\RA_t=0$
if a burst was ending at time $t-\Delta/2$, 
\ag{But in our model the burst of a HVC neuron lasts $dt$, and hence
  also the burst of an RA neuron driven by that HVC neuron. While HVC
  neurons only burst once in a while, RA neurons can be activated by
  different HVC neurons and hence bursting in subsequent time steps of
  size $dt$ -> should we not expect that AC is about zero then between
  different bins of size $dt$? -- Or does this argument use any
  biological results I am not aware of (ie that RA firing is also
  sparse). I feel we can't say anything on AC in the order of
  $\Delta/2$ -- nothing prevents you so far to activate the same RA
  withing $\Delta/2$?}

and we expect that $\AC_\RA(\Delta/2)<0$. Hence, if the RA neuron is activated by a plasticity protocol at time $t - \Delta/2$, i.e.\ $\RA_{t - \Delta/2} = 1$, any HVC synapse must conclude that at time $t$ the RA neuron will not fire anymore, $\langle \RA_t \rangle \approx 0$. Hence, if the presynaptic HVC neuron is activated at time $t-\Delta$, i.e.\ $\HVC_{t - \Delta} = 1$, and if at that time $ \oHVC_{t-\Delta}>0$, long-term depression will be induced according to the plasticity rule
\begin{equation}
\triangle w_{\HVC \to \RA}(t)  = \left( \langle \RA_t \rangle -  \oHVC_{t-\Delta} \right) \, \HVC_{t-\Delta} 
\label{eHVC-RA}
\end{equation}
that is obtained from Eq.\ \ref{HVC-RA} by replacing $\RA_t$ with $\langle \RA_t \rangle$. Similarly, Eq.\ \ref{LMAN-RA} is turned into the postdictive rule  
\begin{equation}
\triangle w_{\LMAN \to \RA}(t)  = \left(  \langle \RA_{t-\Delta} \rangle -  \oLMAN_t \right) \, \LMAN_t \,,
\label{eLMAN-RA}
\end{equation}
such that a spike of the RA neuron at time $t-\Delta/2$, relative to the spike of the presynaptic LMAN neuron at $t$, will induce LTD.

\subsection*{Learning cycle}
%======================================
Given the time discrete simulation in steps of $10\,$ms, $\HVC$ can be seen as a $T \times T$ identity matrix, and $\RA$ and $w_{\HVC \to \RA}$ as $N_\RA \times T$ matrices. When singing from HVC, the RA activity is
\begin{equation}
\RA = w_{\HVC \to \RA} \,\HVC = w_{\HVC \to \RA} \,.
\label{i1}
\end{equation}
Note that the RA activity, and hence these weights, are supposed to be
non-negative. \ag{I have not (yet fully) implemented this constraint of
  nonnegativity for RA activity} 

RA drives the motor area and this generates the SPL$_\psi$ specified by the $2N \times T$ matrix $\psi = M\,\RA$, see Eq.\ \ref{SPLgen}.  HVC also learns to reproduce the delayed tutor song memory in LMAN. Hence, when LMAN is purely driven by HVC, 
\begin{equation}
\LMAN = w_{\HVC \to \LMAN} \,\HVC \,
\label{i2}
\end{equation}
where $w_{\HVC \to \LMAN}$ is a $4N \times T$ matrix because LMAN has
$4N$ neurons. Note $w_{\HVC \to \LMAN}$ is nonnegative by its
definition \eqref{wHVCLMAN} as the auditory components $\phi$ are
positive. This activity is equal to the delayed copy of the tutor
song,  
\begin{equation}
w_{\HVC \to \LMAN} \,\HVC =  \LMAN^\fthr_{-\Delta} \,,
\label{i3}
\end{equation}
where $\LMAN^\fthr_{ -\Delta}(t) = \LMAN^\fthr(t - \Delta)$. When singing from HVC, the auditory signal to LMAN is mixed by the recall of the delayed tutor song as defined in Eq.\ \ref{LMAN}, 
\begin{equation}
\LMAN = \mu \, \LMAN^\fthr_{-\Delta} + (1-\mu) \, \left( \lfloor M\,\RA_{-\Delta} \rfloor ,  \lfloor - M\,\RA_{-\Delta} \rfloor \right) \,,
\label{i4}
\end{equation}
where $M\,\RA$ includes the sine and cosine. The input from LMAN to RA is 
\begin{equation}
\overline\LMAN = w_{\LMAN \to \RA} \,\LMAN  \,,
\label{i5}
\end{equation}
and the weights $w_{\LMAN \to \RA}$ are adapted based on the postdictive learning $\Delta w_{\LMAN \to \RA} = (\RA_{-\Delta} - \overline\LMAN ) \, \LMAN$, see Eq.\ \ref{LMAN-RA} and \ref{eLMAN-RA}, respectively (to simplify the simulations, one may consider the RA activity to be periodically completed to a full cycle). 

Learning of the weights $w_{\HVC \to \RA}$, Eq.\ \ref{HVC-RA}, arises when RA is driven by LMAN. Since HVC represents a delay line, there is a single HVC neuron that is active at a given time and the weights can be set by predictive learning in a single step to
\begin{equation}
w_{\HVC \to \RA} = \RA_{+\Delta} \,,
\label{i6}
\end{equation}
where the shifted $N_\RA \times T$ matrix $\RA_{+\Delta}$, with $\RA_{+\Delta}(t) = \RA(t+\Delta)$, is again periodically completed. But since learning is driven by LMAN, we have $\RA=\lfloor \overline\LMAN \rfloor = \lfloor w_{\LMAN \to \RA} \,\LMAN  \rfloor $, and (\ref{i6}) turns into
\begin{eqnarray}
w_{\HVC \to \RA} =  \left\lfloor w_{\LMAN \to \RA} \,\LMAN_{-\Delta} \right\rfloor  = \left\lfloor w_{\LMAN \to \RA}  \, \LMAN^\fthr  \right\rfloor  \,.
\label{i7}
\end{eqnarray}
For the last step we used (\ref{i4}) and the fact that during the weight copying, when singing is driven by LMAN, there is no auditory feedback, $\mu=1$. From  (\ref{i7}) we conclude that during HVC singing the RA activity is just as it would have been produced by LMAN,
\begin{eqnarray}
\RA = \left\lfloor w_{\LMAN \to \RA}  \, \LMAN^\fthr  \right\rfloor  \,.
\label{i8}
\end{eqnarray}
This RA activity generates the song $M\,\RA$ that feeds back to LMAN with a delay of $\Delta$ via (\ref{i4}),
\begin{equation}
\LMAN = \mu \, \LMAN^\fthr_{-\Delta} + (1-\mu) \, \left( \left\lfloor M\,  \left\lfloor w_{\LMAN \to \RA}  \, \LMAN_{-\Delta}^\fthr  \right\rfloor          \right\rfloor ,  \left\lfloor - M\,     \left\lfloor w_{\LMAN \to \RA}  \, \LMAN_{-\Delta}^\fthr  \right\rfloor        \right \rfloor \right) \,.
\label{i9}
\end{equation}
Going again back to the postdictive learning during HVC singing, $\Delta w_{\LMAN \to \RA} = (\RA_{-\Delta} - \overline\LMAN ) \, \LMAN$, and using (\ref{i5}) and (\ref{i8}), we get 
\begin{eqnarray}
\Delta w_{\LMAN \to \RA} = \left( \left\lfloor w_{\LMAN \to \RA}  \, \LMAN_{-\Delta}^\fthr  \right\rfloor - w_{\LMAN \to \RA} \,\LMAN \right) \, \LMAN  \,,
\label{i10}
\end{eqnarray}
with LMAN given in (\ref{i9}). To understand the iteration of the two equations (\ref{i9}) and (\ref{i10}), we may neglect the nonlinearities and get 
\begin{eqnarray}
\Delta w_{\LMAN \to \RA} & =  & \left( w_{\LMAN \to \RA}  \, \LMAN_{-\Delta}^\fthr  - w_{\LMAN \to \RA} \,\LMAN \right) \, \LMAN  \\
\LMAN & = & \mu \, \LMAN^\fthr_{-\Delta} + (1-\mu) \, M\,  w_{\LMAN \to \RA}  \, \LMAN_{-\Delta}^\fthr \,.
\label{i11}
\end{eqnarray}
Iterating this yields the fixed point $w_{\LMAN \to \RA} = M^{-1}$, as we expected.

\subsection*{Matching the Doupe's plasticity results}
%======================================
According to \citetext{Mehaffey2015} the plasticity windows look as
reproduced in Fig.\ \ref{figDoupe}. We need to match the above
plasticity rules (Eqs \ref{eHVC-RA} and \ref{eLMAN-RA}) to these
curves. Since in the two phases the RA activity is driven by LMAN and
HVC, respectively, we may replace $\langle \RA_t \rangle$ and $\langle
\RA_{t-\Delta} \rangle$ by $\LMAN_t$ and $\HVC_{t-\Delta}$,
respectively. Moreover, due the hyperpolarization of RA soma, we
assume that the plasticity induction machineries in the HVC and LMAN
synapses are both delayed by $\gamma=50$ ms. \ag{Is this an adhoc
  assumption, or supported in the literature?}. The plasticity rules (\ref{eHVC-RA}) and (\ref{eLMAN-RA}) then read as
\begin{eqnarray}
\triangle w_{\HVC \to \RA}(t)  & = & \left( \LMAN_t  -  \oHVC_{t-\gamma-\Delta} \right) \, \HVC_{t-\gamma-\Delta}  \\
\triangle w_{\LMAN \to \RA}(t)  & = & \left(  \HVC_{t-\Delta} -  \oLMAN_{t-\gamma} \right) \, \LMAN_{t-\gamma} \,.   
\label{dwexp}
\end{eqnarray}
This is expected to reproduce the experimental data shown in Fig.\ \ref{figDoupe}.  

\begin{figure}[!ht]  %%%%%%%%%% Figure 1 %%%%%%%%%%%%%%%%%%%%%%%%%%%%%%%%%%%
\vspace{0mm}
\begin{center}
\includegraphics[scale=1.5]{../Figures/Plasticity_Fig_Mehaffey_Doupe_2015.pdf}
%\includegraphics[scale=0.6]{../Figures/ImitationLearning1-F1.pdf}
\end{center}
\vspace{-0.5cm}
\caption{Doupe's plasticity rules for $\HVC_\RA$ and $\LMAN_\RA$  synapses.}
\label{figDoupe}
\end{figure}   %%%%%%%%%% End of Figure 1 %%%%%%%%%%%%%%%%%%%%%%%%%%%%%%%%%


\bibliographystyle{/jneurosci}
\bibliography{/users/senn/Documents/Walter_Physio/Bibfiles/neuroPlasticity,/users/senn/Documents/Walter_Physio/Bibfiles/library}
\end{document} 

